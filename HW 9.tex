\documentclass[fleqn]{article}
\usepackage[margin=1in]{geometry}
\usepackage[utf8]{inputenc}
\usepackage{ulem}
\usepackage{mathtools}
\usepackage{amsmath}
\usepackage{amsthm}
\usepackage{amssymb}
\usepackage{mathabx}

\title{
Math GU4042, Spring 2020 \\
Modern Algebra II, Prof.\ Yihang Zhu \\
HW 9 \\
Notes: 16.11, 17.14-17 \\
Morandi: I.1.\{1, 3, 5, 10, 12\} \\
}

\author{Khyber Sen}
\date{4/9/2020}

\DeclareMathOperator{\Char}{char}
\DeclareMathOperator{\im}{im}

\setcounter{secnumdepth}{0}

\begin{document}
    
    \maketitle
    
    \section{16.11}
    Let $\alpha \in K$ be algebraic over $F$.  Show that the map
    \begin{align}
        ev_{\alpha, F}: F[X] &\to K \\
        f &\mapsto f(\alpha)
    \end{align}
    is a ring homomorphism.  Show that its kernel is the ideal ($\min(\alpha : F)$).  In other words, for $f \in F[X]$, we have $f(\alpha) = 0$ if and only if $f$ is a multiple of $\min(\alpha : F)$ in $F[X]$.  Also, show that the image of $ev_{\alpha, F}$ is $F[\alpha]$, the subring of $K$ generated by $F$ and $\alpha$.
        
        \subsection{Solution}
        To show $ev_{\alpha, F}$ is a ring homomorphism, we just need to show $ev_{\alpha, F}(0) = 0$, $ev_{\alpha, F}(1) = 1$, $ev_{\alpha, F}(a \pm b) = ev_{\alpha, F}(a) \pm ev_{\alpha, F}(b)$, and $ev_{\alpha, F}(ab) = ev_{\alpha, F}(a) ev_{\alpha, F}(b)$.  For $ev_{\alpha, F}(0) = 0$ and $ev_{\alpha, F}(1) = 1$, $0$ and $1$ are both constant functions, so evaluating them at $\alpha$ doesn't change them.  For addition/subtraction, $ev_{\alpha, F}(f \pm g) = (f \pm g)(\alpha) = f(\alpha) \pm g(\alpha) = ev_{\alpha, F}(f) \pm ev_{\alpha, F}(g)$.  And for multiplication, $ev_{\alpha, F}(fg) = (fg)(\alpha) = f(\alpha) g(\alpha) = ev_{\alpha, F}(f) ev_{\alpha, F}(g)$.
        
        To show $\ker(ev_{\alpha, F}) = (\min(\alpha : F))$, we just need to show $\forall$ $f \in F[X]$, $f(\alpha) = 0 \iff f$ is a multiple of $\min(\alpha : F)$.  The $\impliedby$ direction is easy.  By definition, the minimal polynomial of $\alpha$ has a root at $\alpha$, so $\min(\alpha : F)(\alpha) = 0$.  Thus, if $f = g\min(\alpha : F)$, then $f(\alpha) = (g\min(\alpha : F))(0) = g(0) \min(\alpha : F)(0) = g(0) \cdot 0 = 0$.  For the $\implies$ direction, if we factor $f$ into irreducible polynomials, then at least one of these irreducible polynomials must have a root at $\alpha$.  Factoring out the constant of this irreducible polynomial, we are left with a monic, irreducible polynomial with a root at $\alpha$.  The minimal polynomial of $\alpha$ is precisely the unique, monic, irreducible polynomial with a root at $\alpha$, so therefore, $f$ must have $\min(\alpha : F)$ as a factor, i.e., $f$ must be a multiple of $\min(\alpha : F)$.
        
        To show $\im(ev_{\alpha, F}) = F[\alpha]$, we just expand the definitions:
        \begin{align}
            \im(ev_{\alpha, F}) &= \{ev_{\alpha, F}(f) \mid f \in F[X]\} \\
                &= \{f(\alpha) \mid f \in F[X]\} \\
                &= F[\alpha]
        \end{align}
    
    \section{17.14}
    Let $F$ be a field, and let $p \in F[X]$ be a monic irreducible polynomial of degree $d \geq 1$.  Then $L = F[X]/(p)$ is a field extension of $F$.  Show that $1, \overline{X}, \overline{X}^2, ..., \overline{X}^{d - 1}$ is an $F$-basis of $L$.  Conclude that $[L : F] = d$.
        
        \subsection{Solution}
        To show that $B = \{\overline{X}^i \mid i \in 0..d - 1\}$ is an $F$-basis of $L$, we need to show that $B$ is linearly independent and that $B$ spans/generates $L$.  A linear combination of $B$ is just a polynomial in $F[\overline{X}]$, i.e., $L = F[X]/(p)$, of degree $< d$, so clearly $B$ spans $L$.  And since a polynomial is only 0 when all of its coefficients are 0, $B$ must also be linearly independent, so therefore, $B$ is an $F$-basis of $L$.  $[L : F]$ is just the dimension of $L$ as an $F$-vector space, which is just the size of the basis $B$, which is $d$, so therefore, $[L : F] = d$.
    
    \section{17.15}
    Let $K/F$ be a field extension.  Show that $[K : F] = 1$ if and only if $K = F$.
        
        \subsection{Solution}
        If $K = F$, then $K = K^1$ is obviously a 1 dimensional vector space.  For the other direction, if $[K : F] = 1$, then a basis has only 1 vector/element of $K$ that spans $K$.  That is, if $b \in K$ is this base element, then $\forall$ $k \in K$, $\exists$ $f \in F$ s.t.\ $k = fb$.  That is, $f = kb^{-1}$.  Since $k \in K$ and $b^{-1} \in K$, $kb^{-1} \in K$.  Thus, we can only know that $kb^{-1} = f \in F$ if $K = F$.  
    
    \section{17.16}
    Let $K/F$ be a field extension such that $[K : F]$ is a prime number.  Let $\alpha \in K$ be such that $\alpha \notin F$.  Show that $K = F(\alpha)$.  (Hint: use Proposition 17.13.)
        
        \subsection{Solution}
        $F(\alpha)$ is a field that contains $F$, i.e.\ $F(\alpha)/F$ is a field extension.  Since $F(\alpha)$ and $K$ both contain $\alpha$, and $F(\alpha)$ is the minimal field that contains $F$ and $\alpha$, we must have that $F \subset F(\alpha) \subseteq K$.  Assume that $K \neq F(\alpha)$.  Then we have $F \subset F(\alpha) \subset K$.  By Prop. 17.13, $[K : F] = [K : F(\alpha)] [F(\alpha) : F]$.  Since $K \neq F(\alpha)$ and $F(\alpha) \neq F$, neither of these dimensions can be 1.  But since $[K : F]$ is prime, one of these whole number dimensions must be 1.  This is a contradiction, so therefore, $K = F(\alpha)$.
    
    \section{17.17}
    Let $K/F$ be a field extension with $[K : F] = 2$.  Assume that $\Char(F) \neq 2$.
        
        \subsection{(1)}
        Show that there exists $\alpha \in K$ such that $\alpha \notin F$, that $\alpha^2 \in F$, and that $K = F(\alpha)$.  (Hint: First find an element $\beta \in K$ such that $\beta \notin F$.  Argue that $1, \beta, \beta^2$ must be linearly dependent over $F$.  Then try constructing $\alpha$ using $\beta$.  Also note that by the previous exercise, the condition $K = F(\alpha)$ follows from the condition that $\alpha \notin F$, since $[K : F] = 2$ is prime.)
            
            \subsubsection{Solution}
            Let $\beta \in K - F$.  Since $[K : F] = 2$, the dimension of $K$ as an $F$-vector space is 2, so all sets of more than 2 vectors must be linearly dependent.  Thus, $\{1, \beta, \beta^2\}$ must be linearly dependent, meaning $\beta^2 = a + b \beta$ for some $a, b \in F$.  Now let $\alpha = \beta - \frac{1}{2}b$.  Note that we can divide by 2 since $\Char(F) \neq 2$.  Then $\alpha^2 = \left(\beta - \frac{1}{2}b\right)^2 = \beta^2 - 2 \beta \frac{1}{2} b + \frac{1}{4} b^2 = \beta^2 - b \beta + \frac{b^2}{4} = a + \frac{b^2}{4}$.  Since $a, b \in F$, $\alpha^2 \in F$, but $\exists$ $\beta \in K - F$ s.t.\ $\alpha = \beta - \frac{1}{2}b \notin F$.  Thus, we have found an $\alpha \in K$ s.t.\ $\alpha \notin F$ but $\alpha \in F$.  Furthermore, by 17.16, since $\alpha \in K - F$ and $[K : F]$ is prime, $K = F(\alpha)$.
        
        \subsection{(2)}
        Using (1), show that $K/F$ is the splitting field of some $f \in F[X]$ of the form $f = X^2 - u$, for some $u \in F$.
            
            \subsubsection{Solution}
            Let $u = \alpha^2 \in F$, so $f(X) = X^2 - \alpha^2$.  A root of $f$ is $\alpha$, but $\alpha \notin F$, so $f$ is not split over $F$.  But since $\alpha \in K$, $f$ splits over $K$ into $f(X) = (X - \alpha)(X + \alpha)$.  To show $K/F$ is a splitting field of $f$, we also need to show that $F(\alpha, -\alpha) = K$: $F(\alpha, -\alpha) = F(\alpha) = K$, the last part from (1).
    
    \section{I.1.1}
    Let $K$ be a field extension of $F$.  By defining scalar multiplication for $\alpha \in F$ and $a \in K$ by $\alpha \cdot a = \alpha a$, the multiplication in $K$, show that $K$ is an $F$-vector space.
        
        \subsection{Solution}
        To be a vector space, $K$ must be an abelian group, with scalar multiplication $F \times K \to K$ defined over a field $F$ with the properties:
        \begin{align}
            1_F v &= v \\
            0_F v &= 0_K \\
            (c + d) v &= av + dv \\
            c(v + w) &= cv + cw \\
            c(dv) &= (cd) v
        \end{align}
        $K$ is a field, so it's already an abelian group.  Since the scalar multiplication here is just multiplication in $K$, the identity, distributivity, and associativity properties automatically hold.
    
    \section{I.1.3}
    Let $K$ be a field extension of $F$, and let $a \in K$.  Show that the evaluation map $ev_a: F[x] \to K$ given by $ev_a(f(x)) = f(a)$ is a ring and an $F$-vector space homomorphism.
    
    (Only show that map is a $F$-vector space homomorphism.)
        
        \subsection{Solution}
        To show that $ev_a$ is a $F$-vector space linear homomorphism between the $F$-vector spaces $F[x]$ and $K$, we just need to show that $ev_a$ is a group homomorphism and $ev_a(cf) = c ev_a(f)$.  The group homomorphism part is already included under $ev_a$ being a ring homomorphism, so we just need to prove the latter part: $ev_a(cf) = (cf)(a) = cf(a) = c ev_a(f)$.
    
    \section{I.1.5}
    Show that $\mathbb{Q}(\sqrt{5}, \sqrt{7}) = \mathbb{Q}(\sqrt{5} + \sqrt{7})$.
    
    (Here square roots of 5 and 7 are taken in the real numbers.)
        
        \subsection{Solution}
        It's obvious that $\sqrt{5} + \sqrt{7} \in \mathbb{Q}(\sqrt{5}, \sqrt{7})$, so $\mathbb{Q}(\sqrt{5} + \sqrt{7}) \subseteq \mathbb{Q}(\sqrt{5}, \sqrt{7})$.  To show $\mathbb{Q}(\sqrt{5}, \sqrt{7}) = \mathbb{Q}(\sqrt{5} + \sqrt{7})$, then, we just need to show the $\subseteq$ in the other direction, that $\mathbb{Q}(\sqrt{5}, \sqrt{7}) \subseteq \mathbb{Q}(\sqrt{5} + \sqrt{7})$ by showing $\sqrt{5} \in \mathbb{Q}(\sqrt{5} + \sqrt{7})$ and $\sqrt{7} \in \mathbb{Q}(\sqrt{5} + \sqrt{7})$.
        
        Letting $a = 5$ and $b = 7$, we have that $(-a + \sqrt{ab})(\sqrt{a} + \sqrt{b}) = \sqrt{a}\sqrt{b}(\sqrt{a} + \sqrt{b}) - a(\sqrt{a} + \sqrt{b}) = a \sqrt{b} + b \sqrt{a} - a \sqrt{a} - a \sqrt{b} = b \sqrt{a} - a \sqrt{a} = (b - a) \sqrt{a} = (7 - 5) \sqrt{5} = 2 \sqrt{5}$.  Thus, $(-\frac{1}{2} a + \frac{1}{2}\sqrt{ab})(\sqrt{a} + \sqrt{b}) = \sqrt{5}$.  Except for $\sqrt{ab}$, all of these terms are obviously in $\mathbb{Q}(\sqrt{a} + \sqrt{b})$.  For $\sqrt{ab}$, $\frac{1}{2}(\sqrt{a} + \sqrt{b})^2 - a - b) = \frac{1}{2}(a + b + 2\sqrt{ab} - a - b) = \sqrt{ab}$.  Thus, all of the terms are in the field $\mathbb{Q}(\sqrt{a} + \sqrt{b})$, so when we multiply/add them to get $\sqrt{a} = \sqrt{5}$, it's still in the field.  For $\sqrt{b} = \sqrt{7}$, $\sqrt{b} = (\sqrt{a} + \sqrt{b}) - \sqrt{a}$, so $\sqrt{b} \in \mathbb{Q}(\sqrt{a} + \sqrt{b})$, too.  Since $\sqrt{a}, \sqrt{b} \in \mathbb{Q}(\sqrt{a} + \sqrt{b})$, the field generated by them, $\mathbb{Q}(\sqrt{a}, \sqrt{b})$, is a subset of $\mathbb{Q}(\sqrt{a} + \sqrt{b})$.  Since the subset is in both directions, the fields must be equal.
    
    \section{I.1.10}
    If $K$ is a field extension of $F$ and if $a \in K$ such that $[F(a) : F]$ is odd, show that $F(a) = F(a^2)$.  Give an example to show that this can be false if the degree of $F(a)$ over $F$ is even.
        
        \subsection{Solution}
        Since $a^2 \in F(a)$, $F(a^2) \subseteq F(a)$.  Now assume $F(a) \neq F(a^2)$.  Then $F(a^2) \subset F(a)$.  Also, $F \subset F(a^2)$.  By Prop. 17.13, $[F(a) : F] = [F(a) : F(a^2)] [F(a^2) : F]$.  Consider $[F(a) : F(a^2)]$.  It is at most 2 because $a$ is a root of $x^2 - a^2$.  And it is at least 2 because $a \notin F(a^2)$.  Thus, $[F(a) : F(a^2)] = 2$.  This means $[F(a) : F(a^2)] [F(a^2) : F] = 2[F(a^2) : F] = [F(a) : F]$ is even, but it's given that $[F(a) : F]$ is odd, which is a contradiction.  Thus, it must be that $F(a) = F(a^2)$.
        
        When $[F(a) : F]$ is even, however, there is no such contradiction.  For example, consider $K/F = \mathbb{R}/\mathbb{Q}$ and $a = \sqrt{2}$.  $[\mathbb{Q}(\sqrt{2} : \mathbb{F}] = 2$, which is even.  $\mathbb{Q}(\sqrt{2}) \neq \mathbb{Q}(\sqrt{2}^2) = \mathbb{Q}(2)$ because while $\mathbb{Q}(\sqrt{2})$ contains irrational numbers, $\mathbb{Q}(2)$ doesn't.
    
    \section{I.1.12}
    Show that $\mathbb{Q}(\sqrt{2})$ and $\mathbb{Q}(\sqrt{3})$ are not isomorphic as fields but are isomorphic as vector spaces over $\mathbb{Q}$.
        
        \subsection{Solution}
        In general, for a square-free integer $d$, $\mathbb{Q}(\sqrt{d}) = \{a + b\sqrt{d} \mid a, b \in \mathbb{Q}\}$.  Thus, $\{1, \sqrt{d}\}$ is a basis for $\mathbb{Q}(\sqrt{d})$, so $\dim(\mathbb{Q}(\sqrt{d})) = 2$.  2 and 3 are both square-free integers, so $\dim(\mathbb{Q}(\sqrt{2})) = \dim(\mathbb{Q}(\sqrt{3})) = 2$.  Since vector spaces of the same dimension are isomorphic (since they're isomorphic to their coordinates), this means that $\mathbb{Q}(\sqrt{2}) \cong \mathbb{Q}(\sqrt{3})$ are $\mathbb{Q}-$vector spaces.
        
        As fields, however, isomorphism is much more than just having the same dimension.  In general, for distinct square-free integers $a$ and $b$, like 2 and 3, we can show that $\mathbb{Q}(\sqrt{a}) \not\cong \mathbb{Q}(\sqrt{b})$ as fields.  Assume $\mathbb{Q}(\sqrt{a}) \cong \mathbb{Q}(\sqrt{b})$.  Then since $a$ is a square in $\mathbb{Q}(\sqrt{a})$, $a$ must also be a square in $\mathbb{Q}(\sqrt{b})$.  That is, $\exists$ $x + y\sqrt{b} \in \mathbb{Q}(\sqrt{b})$ s.t.\ $a = (x + y\sqrt{b})^2$.  Thus, $(x + y\sqrt{b})^2 = x^2 + 2xy\sqrt{b} + y^2 b = (x^2 + y^2 b) + (2xy) \sqrt{b} = a$.  Since $a \in \mathbb{Q}$, $x^2 + y^2 b = a$ and $2xy = 0$.  Thus, $x$ or $y = 0$.  If $y = 0$, then $a = x^2$.  $x \in \mathbb{Q}$, however, and $a$ is square-free, so by contradiction, $y \neq 0$.  If $x = 0$, then $a = y^2 b$.  Now let $y = \frac{p}{q} \in \mathbb{Q}$ where $\gcd(p, q) = 1$.  Thus, we have $a = \left(\frac{p}{q}\right)^2 b$, so $aq^2 = bp^2$.  Since $a$ and $b$ are square-free, we must have that $|p| = |q| = 1$, which means that $a = b$, which is a contradiction.  Thus, $y \neq 0$, which is another contradiction, meaning $\mathbb{Q}(\sqrt{a}) \not\cong \mathbb{Q}(\sqrt{b})$ as fields.
    
\end{document}
