\documentclass[fleqn]{article}
\usepackage[margin=1in]{geometry}
\usepackage[utf8]{inputenc}
\usepackage[english]{babel}
\usepackage{ulem}
\usepackage{mathtools}
\usepackage{amsmath}
\usepackage{amsthm}
\usepackage{amssymb}
\usepackage{mathabx}

\title{
Math GU4042, Spring 2020 \\
Modern Algebra II, Prof.\ Yihang Zhu \\
Final
}

\author{Khyber Sen}
\date{5/12/2020}

\DeclareMathOperator{\Frac}{Frac}
\DeclareMathOperator{\Gal}{Gal}

\setcounter{secnumdepth}{0}

\begin{document}

    \section{1.}
    (i)
    By Lemma 12.12, we know that $g = X^2 + 1$ is irreducible in $\mathbb{Z}[X]$ iff it has no root in $\mathbb{Z}$.  $X^2 \geq 0$, so $X^2 + 1 > 0$, so $g$ can't have any roots in $\mathbb{Z}$, so therefore, it is irreducible in $\mathbb{Z}[X]$.
    
    $ $ \\
    (ii)
    $f$ is primitive because it's monic, so by Lemma 13.15, $f$ being irreducible in $K[Y]$ implies $f$ is irreducible in $R[Y]$, so we just have to show $f$ is irreducible in $K[Y]$.  We can show this using the Eisenstein criterion.  $f$ is $g$-Eisenstein since $g$ is irreducible in $K = \Frac(R)$ and $g$ divides every coefficient except the leading 1 (also, $g^2$ doesn't divide the constant term).  By the Eisenstein criterion, this means $f$ is irreducible in $K[Y]$, and thus also in $R[Y]$.
    
    \pagebreak
    
    \section{2.}
    For both (i) and (ii), 104 = 105 - 1 = 0 - 1 = -1 since 3 and 5 divide 105.  Thus, we can reduce: $f = X^9 + 104X = X^9 - X = X(X^8 - 1)$.  Thus, there is one 0 root and 8 primitive roots of unity.  The primitive roots of unity are distinct and different from 0, so therefore, there are 9 roots in $K$ for both (i) and (ii).
    
    \pagebreak
    
    \section{3.}
    Let $a = \sqrt[4]{7}$.  First, we know $[K : F] = 4$, since the basis is $\{a^k \mid k \in 0..3\}$.  Thus, to show $K/F$ is Galois, we NTS $|\Gal(K/F)| = [K : F] = 4$ by Theorem 21.2.  We know the Galois group permutes roots, in particular, the roots of $\min(a : F) = x^4 - 7$.  The roots are $\{i^k a \mid k \in 1..4\}$.  Thus, we can create 4 $F$-homomorphisms of $K$ (which are automatically $F$-automorphisms of $K$ and thus in $\Gal(K/F)$ since $K/F$ is finite).  That is, $\Gal(K/F)$ are the homomorphisms generated by mapping $\{a \mapsto i^k a \mid k \in 1..4\}$.
    
    \pagebreak
    
    \section{4.}
    (i)
    For $\deg(p) = 3$, we need to show that $p$ has all distinct roots.  If not, then we'd have the degree be greater than the number of distinct roots, which we know is 3.  Now, we know $K/F$ is Galois, so it is separable, which means $p$ is separable, which means every monic irreducible factor of $p$ is separable.  $p$ is already irreducible, and WLOG, we can treat it as monic, since dividing by the leading coefficient won't change the roots or the degree.  Since monic $p$ is separable, it must have all distinct roots, meaning $\deg(p) = $ the number of distinct roots of $p$ in $K$, which is 3.  Thus, $\deg(p) = 3$.
    
    $ $ \\
    (ii)
    
    
    \pagebreak
    
    \section{5.}
    \begin{align}
        \alpha &= \sqrt{1 + \sqrt{2 + ... \sqrt{7}}} \\
        \alpha^2 &= 1 + \sqrt{2 + ... \sqrt{7}} \\
        \alpha^2 - 1 &= \sqrt{2 + ... \sqrt{7}}
    \end{align}
    Now we just keep repeating this process.  In each step, we start with an $\alpha \in \mathbb{Q}$ and end with an new $\alpha = \alpha^2 - k$, which is still in $\mathbb{Q}$.  Furthermore, the degree is squared in each step.  Thus, after the last step, we are left with $f(\alpha) = 0$, where $f \in \mathbb{Q}[X]$ and $\deg(f) = 2^k$, where $k$ is the number of steps taken.  Since $f = \min(\alpha : \mathbb{Q})$, we now have that $\alpha$ is algebraic in $\mathbb{Q}$ since $f(\alpha) = 0$ and we also know that $\deg(f)$ is a power of 2.
    
    \pagebreak
    
    \section{6.}
    All fields of characteristic 0 have $\mathbb{Q}$ as a subfield.  $[\mathbb{R} : \mathbb{Q}] = 2$ and $[\mathbb{C} : \mathbb{R}] = 2$, so there are no other fields in-between.  In $\mathbb{Q}$ and $\mathbb{R}$, the only automorphism is the identity, but $\mathbb{C}$ has many more automorphisms, so the fact that we have a non-identity automorphism of $K$ here means that $K$ contains $\mathbb{C}$.  In particular, this field automorphism $\sigma$ must fix all of $\mathbb{Q}$, i.e., $\sigma_\mathbb{Q} = id$.  Thus, we need to find an $\alpha \in K - \mathbb{Q}$ s.t.\ $\sigma(\alpha) = -\alpha$.  Note that $\sigma(\alpha^2) = \sigma(\alpha \alpha) = \sigma(\alpha) \sigma(\alpha) = \sigma(\alpha)^2 = (-\alpha)^2 = \alpha^2$.  For example, if $\sigma$ is complex conjugation, an automorphism of $\mathbb{C} \subseteq K$, then $\alpha$ could be any purely imaginary number.
    
    
\end{document}
