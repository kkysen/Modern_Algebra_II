\documentclass[fleqn]{article}
\usepackage[margin=1in]{geometry}
\usepackage[utf8]{inputenc}
\usepackage{ulem}
\usepackage{mathtools}
\usepackage{amsmath}
\usepackage{amsthm}
\usepackage{amssymb}
\usepackage{mathabx}

\title{
Math GU4042, Spring 2020 \\
Modern Algebra II, Prof.\ Yihang Zhu \\
HW 3 \\
D\&F: 7.\{1.14, 3.\{29-31, 33\}, 4.14, 5.\{3-4\}\}
}
\author{Khyber Sen}
\date{2/13/2020}

\DeclareMathOperator{\Char}{char}

\setcounter{secnumdepth}{0}

\begin{document}
    
    \maketitle
    
    \section{7.1.14}
    Let $x$ be a nilpotent element of the commutative ring $R$.
        
        \subsection{(a)}
        Prove that $x$ is either zero or a zero divisor.
            
            \subsubsection{Solution}
            0 is a zero divisor, so we can simplify this to just proving that $x$ is a zero divisor, i.e.\ $\exists$ $y \in R$ s.t.\ $xy = 0$.  Note that $R$ is commutative, so this is equivalent to $yx = 0$ as well.  Since $x$ is nilpotent, $\exists$ $n \in \mathbb{Z}^+$ s.t.\ $x^n = 0$.  Thus, $x x^{n - 1} = 0$.  Thus, $x$ is a zero divisor with $x^{n - 1}$.
        
        \subsection{(b)}
        Prove that $rx$ is nilpotent for all $r \in R$.
            
            \subsubsection{Solution}
            Since $x$ is nilpotent, $\exists$ $n \in \mathbb{Z}^+$ s.t.\ $x^n = 0$.  This same $n$ works to show $rx$ is also nilpotent.  $(rx)^n = r^n x^n$ because the ring is commutative, and then $r^n x^n = r^n \cdot 0 = 0$.
        
        \subsection{(c)}
        Prove that $1 + x$ is a unit in $R$.
            
            \subsubsection{Solution}
            See the proof for part (d).  1 is a unit, so that proof can be applied to show $1 + x$ is a unit in $R$.
        
        \subsection{(d)}
        Deduce that the sum of a nilpotent element and a unit is a unit.
        
            \subsubsection{Solution}
            Let $x$ be a nilpotent element and $y$ be a unit.  We need to prove that $x + y$ is a unit.  Let $n$ be such that $x^n = 0$ since we know $x$ is nilpotent, and because $y$ is a unit, we know $y^{-1}$ $\exists$.  Then consider $z = \sum\limits_{k = 0}^{n - 1} (-1)^k x^k y^{-(k + 1)}$.  We will show that $(x + y)z = 1$ and thus $x + y$ is a unit:
            \begin{align}
                (x + y)z &= (x + y) \sum\limits_{k = 0}^{n - 1} (-1)^k x^k y^{-(k + 1)} \\
                    &= (x + y)(y^{-1} - x y^{-2} + x^2 y^{-3} - x^3 y^{-4} + ... + (-1)^{n - 1} y^{-n} x^n) \\
                    &= (x y^{-1} - x^2 y^{-2} + x^3 y^{-3} - x^4 y^{-4} + ... + x (-1)^{n - 1} y^{-n} x^n) \\
                    &+ (y^0 - x y^{-1} + x^2 y^{-2} - x^3 y^{-3} + ... + y (-1)^{n - 1} y^{-n} x^n) \\
                    &= y^0 + x (-1)^{n - 1} y^{-n} x^n \\
                    &= 1 + (...) 0 \\
                    &= 1
            \end{align}
    
    \section{7.3.29}
    Let $R$ be a commutative ring.  Recall that an element $x \in R$ is nilpotent if $x^n = 0$ for some $n \in \mathbb{Z}^+$.  Prove that the set of nilpotent elements form an ideal---called the \textit{nilradical} of $R$ and denoted by $\mathfrak{N}(R)$.  [Use the Binomial Theorem to show $\mathfrak{N}(R)$ is closed under addition.]
        
        \subsection{Solution}
        First, to show $\mathfrak{N}(R)$ is a subgroup under addition, we need to show it contains 0, is closed under inverses, and is closed under addition.  $\mathfrak{N}(R)$ obviously contains 0, which is trivially nilpotent.  For inverses, if $x \in \mathfrak{N}(R)$, then $x^n = 0$, then $(-x)^n = (-1)^n x^n = (-1)^n \cdot 0 = 0$, so $\mathfrak{N}(R)$ is closed under additive inverses.  For addition, if $x, y \in \mathfrak{N}(R)$, i.e.\ $x^n = 0$ and $y^m = 0$, then we need to show $(x + y) \in \mathfrak{N}(R)$.  Using the binomial theorem, we know
        \begin{align}
            (x + y)^{n + m} &= \sum\limits_{k = 0}^{n + m} \binom{n + m}{k} x^k y^{n + m - k} \\
                &= \sum\limits_{k = 0}^{n - 1} \binom{n + m}{k} x^k y^{n + m - k} + \sum\limits_{k = n}^{n + m} x^k y^{n + m - k} \\
                &= \sum\limits_{k = 0}^{n - 1} \binom{n + m}{k} x^k y^{n + m - k} + \sum\limits_{k = 0}^{m} x^{k + n} y^{n + m - (k + n)} \\
                &= \sum\limits_{k = 0}^{n - 1} \binom{n + m}{k} x^k y^{n + m - k} + \sum\limits_{k = 0}^{m} x^n x^k y^{m - k} \\
                &= y^m \sum\limits_{k = 0}^{n - 1} \binom{n + m}{k} x^k y^{n - k} + x^n \sum\limits_{k = 0}^{m} x^k y^{m - k} \\
                &= 0 \sum\limits_{k = 0}^{n - 1} \binom{n + m}{k} x^k y^{n - k} + 0 \sum\limits_{k = 0}^{m} x^k y^{m - k} \\
                &= 0 + 0 = 0
        \end{align}
        
        For the absorbing property, we already know from 7.1.14(b) if $x$ is nilpotent and $r \in R$, then $rx$ is also nilpotent, so thus $\mathfrak{N}(R)$ is an ideal.
    
    \section{7.3.30}
    Prove that if $R$ is a commutative ring and $\mathfrak{N}(R)$ is its nilradical then zero is the only nilpotent element of $R/\mathfrak{N}(R)$ i.e., prove that $\mathfrak{N}({R/\mathfrak{N}(R)}) = 0$.
        
        \subsection{Solution}
        $\forall$ $\overline{x} \in \mathfrak{N}(R/\mathfrak{N}(R))$, where the bar denotes passage to the quotient ring, $\overline{x}^n = \overline{0}$ since $\overline{x}$ is nilpotent.  Since $\overline{x}^n = \overline{0}$, $\overline{x^n} = 0$, so $x^n \in \mathfrak{N}(R)$.  Therefore, $\exists$ $m \in \mathbb{Z}^+$ s.t.\ $(x^n)^m = 0$, so $x^{nm} = 0$, meaning $x \in \mathfrak{N}(R)$.  Therefore, $\overline{x} = \overline{0}$, so $\mathfrak{N}(R/\mathfrak{N}(R)) = 0 = \{\overline{0}\}$.
    
    \section{7.3.31}
    Prove that the elements $\begin{pmatrix}
        0 & 1 \\
        0 & 0
    \end{pmatrix}$ and $\begin{pmatrix}
        0 & 0 \\
        1 & 0
    \end{pmatrix}$ are nilpotent elements of $M_2(\mathbb{Z})$ whose sum is not nilpotent (note that these two matrices do not commute).  Deduce that the set of nilpotent elements in the noncommutative ring $M_2(\mathbb{Z})$ is not an ideal.
        
        \subsection{Solution}
        The matrices are both nilpotent (when squared):
        \begin{align}
            \begin{bmatrix}
                0 & 1 \\
                0 & 0
            \end{bmatrix} \begin{bmatrix}
                0 & 1 \\
                0 & 0
            \end{bmatrix} &= \begin{bmatrix}
                0 & 0 \\
                0 & 0
            \end{bmatrix} = 0 \\
            \begin{bmatrix}
                0 & 0 \\
                1 & 0
            \end{bmatrix} \begin{bmatrix}
                0 & 0 \\
                1 & 0
            \end{bmatrix} &= \begin{bmatrix}
                0 & 0 \\
                0 & 0
            \end{bmatrix} = 0
        \end{align}
        To show that $\mathfrak{N}(M_2(\mathbb{Z}))$ is not an ideal, we can show it doesn't follow the absorbing property if we look at these two given matrices.  Their product, $\begin{bmatrix}
            0 & 1 \\
            0 & 0
        \end{bmatrix} \begin{bmatrix}
            0 & 0 \\
            1 & 0
        \end{bmatrix} = \begin{bmatrix}
            1 & 0 \\
            0 & 0
        \end{bmatrix}$, is diagonal, so we can compute its integral powers easily: $\begin{bmatrix}
            1 & 0 \\
            0 & 0
        \end{bmatrix}^n = \begin{bmatrix}
            1^n & 0 \\
            0 & 0^n
        \end{bmatrix} = \begin{bmatrix}
            1 & 0 \\
            0 & 0
        \end{bmatrix} \neq 0$.  Thus their product is not nilpotent, meaning the nilradical of such a noncommutative ring isn't ideal.

    \section{7.3.33}
    Assume $R$ is commutative.  Let $p(x) = a_n x^n + a_{n - 1} x^{n - 1} + ... + a_1 x + a_0$ be an element of the polynomial ring $R[x]$.
        
        \subsection{(a)}
        Prove that $p(x)$ is a unit in $R[x]$ if and only if $a_0$ is a unit and $a_1, a_2, ..., a_n$ are nilpotent in $R$.
            
            \subsubsection{Solution}
            For the $\implies$ direction, $p(x)$ is a unit, so we know $\exists$ $p^{-1}(x)$ s.t.\ $p(x) p^{-1}(x) = 1$.  Let $p(x) = \sum\limits_{k = 0}^{n} a_k x^k$, $p^{-1}(x) = \sum\limits_{k = 0}^{n} b_k x^k$.  For $p(x)p^{-1}(x)$ to be 1, the constant term of the product must be 1.  This constant term is given by $a_0 b_0$.  Thus, $a_0 b_0 = 1$, meaning $a_0$ is a unit and ${a_0}^{-1} = b_0$.
            
            Furthermore, for $p(x)p^{-1}(x)$ to be 1, all the other non-constant coefficients must be 0, so $c_k = \sum\limits_{i = 0}^{k} a_i b_{k - i} = 0$ $\forall$ $k \in 1..n$.  If we multiply $c_k$ by ${a_n}^{n - k}$ and substitute between different $c_k$, we then get $0 = c_k {a_n}^{n - k} = {a_n}^{n - k + 1} b_{n - k}$.  Thus, ${a_n}^{n + 1} b_0 = 0$.  Since $b_0$ is a unit, we can divide it out, giving us ${a_n}^{n + 1} = 0$, meaning $a_n$ is nilpotent.  Since the nilradical is an ideal, this means that $a_n x^n$ and $p(x) - a_n x^n$ are also nilpotent.  Now we've reduced $p(x)$ to a lower degree, and we can now apply the same argument to keep showing $a_k$ is nilpotent $\forall$ $k \in n..1$.
            
            For the $\impliedby$ direction, we know $p(x) - a_0$ is nilpotent by part (b) below, since it's constant term is 0, which is nilpotent, and all the other coefficients, $a_1, ..., a_n$ are nilpotent.  From problem 7.1.14(d), we know the sum of a nilpotent element and a unit is a unit, and we have $p(x) - a_0$ is nilpotent and $a_0$ is a unit, so their sum $p(x) - a_0 + a_0 = p(x)$ is also a unit.
        
        
        \subsection{(b)}
        Prove that $p(x)$ is nilpotent in $R[x]$ if and only if $a_0, a_1, ..., a_n$ are nilpotent elements of $R$.
        
            \subsubsection{Solution}
            For the $\implies$ direction, if $p(x)$ is nilpotent, then $\exists$ $m$ s.t.\ $p(x)^{m_0} = 0$.  In particular, this means that ${p(x)^{m_0}}_0 = 0$, so we know ${a_0}^{m_0} = 0$, meaning $a_0$ is nilpotent, both in $R$ and $R[x]$.  Since both $a_0$ and $p(x)$ are nilpotent in $R[x]$, their difference, $p(x) - a_0$, is also nilpotent.  $p(x) - a_0$ has a 0 constant term, so we can now apply the argument from before to the next term, i.e.\ ${(p(x) - a_0)^{m_1}}_1 = 0$, so ${a_1}^{m_1}$, meaning $a_1$ is also nilpotent.  Repeating this argument, we see that all $a_0, ..., a_n$ are all nilpotent.
            
            For the $\impliedby$ direction, if $a_0, ..., a_n$ are all nilpotent in $R$, then they are also all nilpotent in $R[x]$.  Furthermore, for each nilpotent $a_i$, $a_i x^i$ is also nilpotent since $\mathfrak{N}(R[x])$ is an ideal.  Summing these nilpotent terms together, we get $p(x)$, which is also nilpotent, again since $\mathfrak{N}(R[x])$ is an ideal.
    
    \section{7.4.14}
    Assume $R$ is commutative.  Let $x$ be an indeterminate, let $f(x)$ be a monic polynomial in $R[x]$ of degree $n \geq 1$ and use the bar notation to denote passage to the quotient ring $R[x]/(f(x))$.
        
        \subsection{(a)}
        Show that every element of $R[x]/(f(x))$ is of the form $\overline{p(x)}$ for some polynomial $p(x) \in R[x]$ of degree less than $n$, i.e., 
        \begin{align}
            R[x]/(f(x)) &= \{\overline{a_0} + \overline{a_1 x} + ... + \overline{a_{n - 1} x^{n - 1}} \mid a_0, a_1, ..., a_{n - 1} \in R\}
        \end{align}
        [If $f(x) = x^n + b_{n - 1} x^{n - 1} + ... + b_0$ then $\overline{x^n} = \overline{-(b_{n - 1} x^{n - 1} + ... + b_0)}$.  Use this to reduce powers of $\overline{x}$ in the quotient ring.]
            
            \subsubsection{Solution}
            It's obvious that every element of $R[x]/(f(x))$ is of the form $\overline{p(x)}$ for some polynomial $p(x) \in R[x]$; we just have to show the degree of $p(x)$ is $< n$.  Let $m$ be the degree of $p(x)$. If $m < n$, then the proof is done, and if $m \geq n$, we just have to show $\overline{p(x)} = \overline{q(x)}$ where $q(x)$ is of degree $< n$.  Since $\overline{f(x)} = \overline{0}$, $\overline{\sum\limits_{k = 0}^{n} a_k x^k} = \overline{0}$ (where $a_n = 1$), so $\overline{x^n} = \overline{-\sum\limits_{k = 0}^{n - 1} a_k x^k}$.  This means for every $x^k$ term in $p(x)$ where $k \geq n$, we can factor out an $\overline{x^n}$ and replace it with $\overline{-\sum\limits_{k = 0}^{n - 1} a_k x^k}$, which has degree $< n$.  This allows us to find an equivalent polynomial $q(x)$ with degree $< n$ but still have $\overline{p(x)} = \overline{q(x)}$.
        
        \subsection{(b)}
        Prove that if $p(x)$ and $q(x)$ are distinct polynomials in $R[x]$ which are both of degree less than $n$, then $\overline{p(x)} \neq \overline{q(x)}$.  [Otherwise $p(x) - q(x)$ is an $R[x]$-multiple of the monic polynomial $f(x)$.]
            
            \subsubsection{Solution}
            Assume $\overline{p(x)} = \overline{q(x)}$.  Then $\overline{p(x) - q(x)} = \overline{0}$, so $p(x) - q(x) = r(x)f(x)$.  Since $p(x) - q(x)$ are both of degree $< n$, $p(x) - q(x)$ is also of degree of $< n$.  However, $f(x)$ is of degree $n$, so $r(x)$ must be 0 to cancel out the higher degree from $f(x)$.  Thus, we have $p(x) - q(x)$, meaning $p(x) = q(x)$, which is a contradiction.  Thus, we must have $\overline{p(x)} \neq \overline{q(x)}$.
        
        \subsection{(c)}
        If $f(x) = a(x) b(x)$ where both $a(x)$ and $b(x)$ have degree less than $n$, prove that $\overline{a(x)}$ is a zero divisor in $R[x]/(f(x))$.
            
            \subsubsection{Solution}
            Since $a(x)$ and $b(x)$ both have degree $< n$, we know $\overline{a(x)} \neq 0$ and $\overline{b(x)} \neq 0$.  But we also have $f(x) = a(x)b(x)$, so $\overline{a(x)}\overline{b(x)} = \overline{a(x)b(x)} = \overline{f(x)} = \overline{0}$, which means $\overline{a(x)}$ and $\overline{b(x)}$ are both zero divisors in $R[x]/(f(x))$.
        
        \subsection{(d)}
        If $f(x) = x^n - a$ for some nilpotent element $a \in R$, prove that $\overline{x}$ is nilpotent in $R[x]/(f(x))$.
            
            \subsubsection{Solution}
            Firstly, $\overline{x^n - a} = \overline{f(x)} = \overline{0}$, so $\overline{x^n} = \overline{a}$.  Then, since $a$ is nilpotent in $R$, $\exists$ $m$ s.t.\ $a^m = 0$.  Therefore, $\overline{x}^{nm} = \overline{x^n}^m = \overline{a}^m = \overline{a^m} = \overline{0}$, meaning $\overline{x}$ is nilpotent in $R[x]/(f(x))$.
        
        \subsection{(e)}
        Let $p$ be a prime, assume $R = \mathbb{F}_p$ and $f(x) = x^p - a$ for some $a \in \mathbb{F}_p$.  Prove that $\overline{x - a}$ is nilpotent in $R[x]/(f(x))$.
            
            \subsubsection{Solution}
            Firstly, $\overline{x^p - a} = \overline{f(x)} = \overline{0}$, so $\overline{x^p} = \overline{a}$.  Then
            \begin{align}
                \overline{x - a}^p &= \overline{(x - a)^p} \\
                    &= \overline{\sum\limits_{k = 0}^{p} \binom{p}{k} x^k (-a)^{p - k}} \\
                    &= \overline{x^p + (-a)^p + \sum\limits_{k = 1}^{p - 1} \binom{p}{k} x^p (-a)^{p - k}} \\
                    &= \overline{x^p + (-a)^p} \\
                    &= \overline{x^p + (-a)} \\
                    &= \overline{x^p - a} \\
                    &= \overline{a} - \overline{a} \\
                    &= \overline{0}
            \end{align}
            $\sum\limits_{k = 1}^{p - 1} \binom{p}{k} x^p (-a)^{p - k} = 0$ is true because $\binom{p}{k}$ is a multiple of $p$ $\forall k \in 1..p - 1$, and since $\Char(\mathbb{F}_p) = p$, multiplying by $p$ is multiplying by 0.  Furthermore, for prime $p$ in $\mathbb{F}_p$, we know $a^p = a$, which explains the next step.
    
    \section{7.5.3}
    Let $F$ be a field.  Prove that $F$ contains a unique smallest subfield $F_0$ and that $F_0$ is isomorphic to either $\mathbb{Q}$ or $\mathbb{Z}/p\mathbb{Z}$ for some prime $p$ ($F_0$ is called the \textit{prime subfield} of $F$).
        
        \subsection{Solution}
        Since the intersection of subgroups of some group is also a subgroup of that group, we know that the intersection of subfields is also a subfield since a field is just an additive and multiplicative group.  Now let $F_0$ be the intersection, which is unique, of all subfields of $F$.  Then it is contained in every subfield, so it is the smallest subfield.
        
        Now, since $F_0$ is a field and thus a domain, $\Char(F_0)$ is either 0 are a prime $p$.  By closure as an additive group, $\sum\limits_{i = 1}^{n} \in F_0, n \in \mathbb{Z}$.  If $\Char(F_0)$, this generates all of $\mathbb{Z}$; otherwise, it generates $\mathbb{Z}/p\mathbb{Z}$, meaning $F_0$ contains either $\mathbb{Z}$ or $\mathbb{Z}/p\mathbb{Z}$.  The latter is a field, and since $F_0$ is the smallest subfield, if $\Char(F_0) \neq 0$, then $\mathbb{Z}/p\mathbb{Z} \subseteq F_0$, so $F_0 \cong \mathbb{Z}/p\mathbb{Z}$.  If $\Char(F_0) = 0$, then $\mathbb{Z} \subseteq F_0$.  $\mathbb{Z}$ isn't a field, however, so we can't make that same argument.  However, since $F_0$ contains $\mathbb{Z}$, it must also contain $\mathbb{Q}$ by closure as a multiplicative group, because $\forall$ $x, y \in \mathbb{Z}$ s.t.\ $y \neq 0$, $\frac{x}{y} \in F_0$ by closure and $\frac{x}{y} \in \mathbb{Q}$.  Since we now know $\mathbb{Q} \subseteq F_0$ and $\mathbb{Q}$ is a field, we can apply the same argument from before to conclude that $F_0 \cong \mathbb{Q}$.
    
    \section{7.5.4}
    Prove that any subfield of $\mathbb{R}$ must contain $\mathbb{Q}$.
        
        \subsection{Solution}
        Let $F$ be a subfield of $\mathbb{R}$.  As a field, it must contain 1.  By closure as an additive subgroup and as a subfield of $\mathbb{R}$, it must also contain all of $\mathbb{Z}$, i.e.\ if $x = \sum\limits_{i = 1}^{n} 1, n \in \mathbb{Z}$, then $x \in F$ by closure, and this generates all of $\mathbb{Z}$ because $F$ is also characteristic 0 like $\mathbb{R}$.  Furthermore, $\forall$ $x, y \in \mathbb{Z}$ s.t.\ $y \neq 0$, $\frac{x}{y} \in F$ by closure as a multiplicative subgroup.  $\frac{x}{y} \in \mathbb{Q}$, so $F$ must therefore contain $\mathbb{Q}$.
    
\end{document}
