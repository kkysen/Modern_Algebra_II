\documentclass[fleqn]{article}
\usepackage[margin=1in]{geometry}
\usepackage[utf8]{inputenc}
\usepackage{ulem}
\usepackage{mathtools}
\usepackage{amsmath}
\usepackage{amsthm}
\usepackage{amssymb}
\usepackage{mathabx}

\title{
Math GU4042, Spring 2020 \\
Modern Algebra II, Prof.\ Yihang Zhu \\
HW 3
}
\author{Khyber Sen}
\date{2/13/2019}

\setcounter{secnumdepth}{0}

\begin{document}
    
    \maketitle
    
    \section{7.1.14}
    Let $x$ be a nilpotent element of the commutative ring $R$.
        
        \subsection{(a)}
        Prove that $x$ is either zero or a zero divisor.
            
            \subsubsection{Solution}
            0 is a zero divisor, so we can simplify this to just proving that $x$ is a zero divisor, i.e.\ $\exists$ $y \in R$ s.t.\ $xy = 0$.  Note that $R$ is commutative, so this is equivalent to $yx = 0$ as well.  Since $x$ is nilpotent, $\exists$ $n \in \mathbb{Z}^+$ s.t.\ $x^n = 0$.  Thus, $x x^{n - 1} = 0$.  Thus, $x$ is a zero divisor with $x^{n - 1}$.
        
        \subsection{(b)}
        Prove that $rx$ is nilpotent for all $r \in R$.
            
            \subsubsection{Solution}
            Since $x$ is nilpotent, $\exists$ $n \in \mathbb{Z}^+$ s.t.\ $x^n = 0$.  This same $n$ works to show $rx$ is also nilpotent.  $(rx)^n = r^n x^n$ because the ring is commutative, and then $r^n x^n = r^n \cdot 0 = 0$.
        
        \subsection{(c)}
        Prove that $1 + x$ is a unit in $R$.
            
            \subsubsection{Solution}
            See the proof for part (d).  1 is a unit, so that proof can be applied to show $1 + x$ is a unit in $R$.
        
        \subsection{(d)}
        Deduce that the sum of a nilpotent element and a unit is a unit.
        
            \subsubsection{Solution}
            Let $x$ be a nilpotent element and $y$ be a unit.  We need to prove that $x + y$ is a unit.  Let $n$ be such that $x^n = 0$ since we know $x$ is nilpotent, and because $y$ is a unit, we know $y^{-1}$ $\exists$.  Then consider $z = \sum\limits_{k = 0}^{n - 1} (-1)^k x^k y^{-(k + 1)}$.  We will show that $(x + y)z = 1$ and thus $x + y$ is a unit:
            \begin{align}
                (x + y)z &= (x + y) \sum\limits_{k = 0}^{n - 1} (-1)^k x^k y^{-(k + 1)} \\
                    &= (x + y)(y^{-1} - x y^{-2} + x^2 y^{-3} - x^3 y^{-4} + ... + (-1)^{n - 1} y^{-n} x^n) \\
                    &= (x y^{-1} - x^2 y^{-2} + x^3 y^{-3} - x^4 y^{-4} + ... + x (-1)^{n - 1} y^{-n} x^n) \\
                    &+ (y^0 - x y^{-1} + x^2 y^{-2} - x^3 y^{-3} + ... + y (-1)^{n - 1} y^{-n} x^n) \\
                    &= y^0 + x (-1)^{n - 1} y^{-n} x^n \\
                    &= 1 + (...) 0 \\
                    &= 1
            \end{align}
    
    \section{7.3.29}
    Let $R$ be a commutative ring.  Recall that an element $x \in R$ is nilpotent if $x^n = 0$ for some $n \in \mathbb{Z}^+$.  Prove that the set of nilpotent elements form an ideal---called the \textit{nilradical} of $R$ and denoted by $\mathfrak{N}(R)$.  [Use the Binomial Theorem to show $\mathfrak{N}(R)$ is closed under addition.]
        
        \subsection{Solution}
        First, to show $\mathfrak{N}(R)$ is a subgroup under addition, we need to show it contains 0, is closed under inverses, and is closed under addition.  $\mathfrak{N}(R)$ obviously contains 0, which is trivially nilpotent.  For inverses, if $x \in \mathfrak{N}(R)$, then $x^n = 0$, then $(-x)^n = (-1)^n x^n = (-1)^n \cdot 0 = 0$, so $\mathfrak{N}(R)$ is closed under additive inverses.  For addition, if $x, y \in \mathfrak{N}(R)$, i.e.\ $x^n = 0$ and $y^m = 0$, then we need to show $(x + y) \in \mathfrak{N}(R)$.  Using the binomial theorem, we know
        \begin{align}
            (x + y)^{n + m} &= \sum\limits_{k = 0}^{n + m} \binom{n + m}{k} x^k y^{n + m - k} \\
                &= \sum\limits_{k = 0}^{n - 1} \binom{n + m}{k} x^k y^{n + m - k} + \sum\limits_{k = n}^{n + m} x^k y^{n + m - k} \\
                &= \sum\limits_{k = 0}^{n - 1} \binom{n + m}{k} x^k y^{n + m - k} + \sum\limits_{k = 0}^{m} x^{k + n} y^{n + m - (k + n)} \\
                &= \sum\limits_{k = 0}^{n - 1} \binom{n + m}{k} x^k y^{n + m - k} + \sum\limits_{k = 0}^{m} x^n x^k y^{m - k} \\
                &= y^m \sum\limits_{k = 0}^{n - 1} \binom{n + m}{k} x^k y^{n - k} + x^n \sum\limits_{k = 0}^{m} x^k y^{m - k} \\
                &= 0 \sum\limits_{k = 0}^{n - 1} \binom{n + m}{k} x^k y^{n - k} + 0 \sum\limits_{k = 0}^{m} x^k y^{m - k} \\
                &= 0 + 0 = 0
        \end{align}
        
        For the absorbing property, we already know from 7.1.14(b) if $x$ is nilpotent and $r \in R$, then $rx$ is also nilpotent, so thus $\mathfrak{N}(R)$ is an ideal.
    
    \section{7.3.30}
    Prove that if $R$ is a commutative ring and $\mathfrak{N}(R)$ is its nilradical then zero is the only nilpotent element of $R/\mathfrak{N}(R)$ i.e., prove that $\mathfrak{N}({R/\mathfrak{N}(R)}) = 0$.
        
        \subsection{Solution}
        0 is obviously always nilpotent, so we just have to show if $x \in R/\mathfrak{N}$ and $x \neq 0$, then $x$ is not nilpotent, i.e.\ $x^n \neq 0$ $\forall$ $n \in \mathbb{Z}^+$.
        
        TODO
    
    \section{7.3.31}
    Prove that the elements $\begin{pmatrix}
        0 & 1 \\
        0 & 0
    \end{pmatrix}$ and $\begin{pmatrix}
        0 & 0 \\
        1 & 0
    \end{pmatrix}$ are nilpotent elements of $M_2(\mathbb{Z})$ whose sum is not nilpotent (note that these two matrices do not commute).  Deduce that the set of nilpotent elements in the noncommutative ring $M_2(\mathbb{Z})$ is not an ideal.
        
        \subsection{Solution}
        The matrices are both nilpotent (when squared):
        \begin{align}
            \begin{bmatrix}
                0 & 1 \\
                0 & 0
            \end{bmatrix} \begin{bmatrix}
                0 & 1 \\
                0 & 0
            \end{bmatrix} &= \begin{bmatrix}
                0 & 0 \\
                0 & 0
            \end{bmatrix} = 0 \\
            \begin{bmatrix}
                0 & 0 \\
                1 & 0
            \end{bmatrix} \begin{bmatrix}
                0 & 0 \\
                1 & 0
            \end{bmatrix} &= \begin{bmatrix}
                0 & 0 \\
                0 & 0
            \end{bmatrix} = 0
        \end{align}
        To show that $\mathfrak{N}(M_2(\mathbb{Z}))$ is not an ideal, we can show it doesn't follow the absorbing property if we look at these two given matrices.  Their product, $\begin{bmatrix}
            0 & 1 \\
            0 & 0
        \end{bmatrix} \begin{bmatrix}
            0 & 0 \\
            1 & 0
        \end{bmatrix} = \begin{bmatrix}
            1 & 0 \\
            0 & 0
        \end{bmatrix}$, is diagonal, so we can compute its integral powers easily: $\begin{bmatrix}
            1 & 0 \\
            0 & 0
        \end{bmatrix}^n = \begin{bmatrix}
            1^n & 0 \\
            0 & 0^n
        \end{bmatrix} = \begin{bmatrix}
            1 & 0 \\
            0 & 0
        \end{bmatrix} \neq 0$.  Thus their product is not nilpotent, meaning the nilradical of such a noncommutative ring isn't ideal.
    
    \section{7.3.33}
    Assume $R$ is commutative.  Let $p(x) = a_n x^n + a_{n - 1} x^{n - 1} + ... + a_1 x + a_0$ be an element of the polynomial ring $R[x]$.
        
        \subsection{(a)}
        Prove that $p(x)$ is a unit in $R[x]$ if and only if $a_0$ is a unit and $a_1, a_2, ..., a_n$ are nilpotent in $R$.
            
            \subsubsection{Solution}
            Let $p(x) = \sum\limits_{k = 0}^{n} a_k x^k$, $q(x) = \sum\limits_{k = 0}^{n} b_k x^k$, and $r(x) = \sum\limits_{k = 0}^{n} c_k x^k$.  We want to show that $p(x)$ is a unit, i.e.\ $p(x)q(x) = r(x) = 1 \iff a_0 \in R^\times$ and $a_1, ..., a_n \in \mathfrak{N}(R)$.  For the $\impliedby$ direction, given $a_0$ is a unit and $a_1, ..., a_n$ are nilpotent, we need to show $c_0 = 1$ and $c_k = 0$ $\forall$ $k \neq 0$.  $c_k = \sum\limits_{j = 0}^{k} a_j b_{n - j}$.  Let $b_0 = {a_0}^{-1}$, which we know exists since $a_0$ is a unit.  Then $c_0 = \sum\limits_{j = 0}^0 a_j b_{0 - j} = a_0 b_0 = a_0 {a_0}^{-1} = 1$.  Then $\forall$ $k \neq 0$, let $b_k = -{a_0}^{-1} \sum\limits_{j = 1}^{k} a_j b_{k - j}$.  Then,
            \begin{align}
                c_k &= \sum\limits_{j = 0}^{k} a_j b_{k - j} \\
                    &= \sum\limits_{j = 0}^{k} a_j \left(-{a_0}^{-1} \sum\limits_{j = 1}^{k - j} a_j b_{(k - j) - j}\right) \\
                    &= a_0 \left(-{a_0}^{-1} \sum\limits_{j = 1}^{k} a_j b_{k - j}\right) + \sum\limits_{j = 1}^{k} a_j \left(-{a_0}^{-1} \sum\limits_{j = 1}^{k - j} a_j b_{(k - j) - j}\right) \\
                    &= -\sum\limits_{j = 1}^{k} a_j b_{k - j} + \sum\limits_{j = 1}^{k} a_j \left(-{a_0}^{-1} \sum\limits_{j = 1}^{k - j} a_j b_{(k - j) - j}\right) \\
                    &= -\sum\limits_{j = 1}^{k} a_j b_{k - j} + \sum\limits_{j = 1}^{k} a_j b_{k - j} \because{} b_{k - j} = -{a_0}^{-1} \sum\limits_{j = 1}^{k - j} a_j b_{(k - j) - j} \text{ by definition}\\
                    &= 0
            \end{align}
            TODO why do $a_1, ..., a_n$ have to be nilpotent
        
        
        \subsection{(b)}
        Prove that $p(x)$ is nilpotent in $R[x]$ if and only if $a_0, a_1, ..., a_n$ are nilpotent elements of $R$.
        
            \subsubsection{Solution}
            
    
    \section{7.4.14}
    Assume $R$ is commutative.  Let $x$ be an indeterminate, let $f(x)$ be a monic polynomial in $R[x]$ of degree $n \geq 1$ and use the bar notation to denote passage to the quotient ring $R[x]/(f(x))$.
        
        \subsection{(a)}
        Show that every element of $R[x]/(f(x))$ is of the form $\overline{p(x)}$ for some polynomial $p(x) \in R[x]$ of degree less than $n$, i.e., 
        \begin{align}
            R[x]/(f(x)) &= \{\overline{a_0} + \overline{a_1 x} + ... + \overline{a_{n - 1} x^{n - 1}} \mid a_0, a_1, ..., a_{n - 1} \in R\}
        \end{align}
        [If $f(x) = x^n + b_{n - 1} x^{n - 1} + ... + b_0$ then $\overline{x^n} = \overline{-(b_{n - 1} x^{n - 1} + ... + b_0)}$.  Use this to reduce powers of $\overline{x}$ in the quotient ring.]
            
            \subsubsection{Solution}
            
        
        \subsection{(b)}
        Prove that if $p(x)$ and $q(x)$ are distinct polynomials in $R[x]$ which are both of degree less than $n$, then $\overline{p(x)} \neq \overline{q(x)}$.  [Otherwise $p(x) - q(x)$ is an $R[x]$-multiple of the monic polynomial $f(x)$.]
            
            \subsubsection{Solution}
            
        
        \subsection{(c)}
        If $f(x) = a(x) b(x)$ where both $a(x)$ and $b(x)$ have degree less than $n$, prove that $\overline{a(x)}$ is a zero divisor in $R[x]/(f(x))$.
            
            \subsubsection{Solution}
            
        
        \subsection{(d)}
        If $f(x) = x^n - a$ for some nilpotent element $a \in R$, prove that $\overline{x}$ is nilpotent in $R[x]/(f(x))$.
            
            \subsubsection{Solution}
            
        
        \subsection{(e)}
        Let $p$ be a prime, assume $R = \mathbb{F}_p$ and $f(x) = x^p - a$ for some $a \in \mathbb{F}_p$.  Prove that $\overline{x - a}$ is nilpotent in $R[x]/(f(x))$.
            
            \subsubsection{Solution}
            
    
    \section{7.5.3}
    Let $F$ be a field.  Prove that $F$ contains a unique smallest subfield $F_0$ and that $F_0$ is isomorphic to either $\mathbb{Q}$ or $\mathbb{Z}/p\mathbb{Z}$ for some prime $p$ ($F_0$ is called the \textit{prime subfield} of $F$).
        
        \subsection{Solution}
        
    
    \section{7.5.4}
    Prove that any subfield of $\mathbb{R}$ must contain $\mathbb{Q}$.
        
        \subsection{Solution}
        
    
\end{document}
