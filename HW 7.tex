\documentclass[fleqn]{article}
\usepackage[margin=1in]{geometry}
\usepackage[utf8]{inputenc}
\usepackage{ulem}
\usepackage{mathtools}
\usepackage{amsmath}
\usepackage{amsthm}
\usepackage{amssymb}
\usepackage{mathabx}

\title{
Math GU4042, Spring 2020 \\
Modern Algebra II, Prof.\ Yihang Zhu \\
HW 7 \\
D\&F: 9.2.12, 9.3.\{1-4\} \\
Notes: 13.9 \\
}
\author{Khyber Sen}
\date{3/19/2020}

\newcommand{\Mod}[1]{\ (\mathrm{mod}\ #1)}

\setcounter{secnumdepth}{0}

\begin{document}
    
    \maketitle
    
    \section{9.2.12}
    Let $F[x, y_1, y_2, ...]$ be the polynomial ring in the infinite set of variables $x, y_1, y_2, ...$ over the field $F$, and let $I$ be the ideal $\left(x - {y_1}^2, y_1 - {y_2}^2, ..., y_i - {y_{i + 1}}^2, ...\right)$ in this ring.  Define $R$ to be the ring $F[x, y_1, y_2, ...]/I$, so that in $R$ the square of each $y_{i + 1}$ is $y_i$ and ${y_1}^2 = x$ modulo $I$, i.e., $x$ has a $2^i$th root, for every $i$.  Denote the image of $y_i$ in $R$ as $x^{\frac{1}{2^i}}$.  Let $R_n$ be the subring of $R$ generated by $F$ and $x^{\frac{1}{2^n}}$.
        
        \subsection{a.}
        Prove that $R_1 \subseteq R_2 \subseteq ...$ and that $R$ is the union of all $R_n$, i.e., $R = \bigcup\limits_{n = 1}^{\infty} R_n$.
            
            \subsubsection{Solution}
            
        
        \subsection{b.}
        Prove that $R_n$ is isomorphic to a polynomial ring in one variable over $F$, so that $R_n$ is a P.I.D.  Deduce that $R$ is a Bezout Domain.  [First show that the ring $S_n = F[x, y_1, ..., y_n]/\left(x - {y_1}^2, y_1 - {y_2}^2, ..., y_{n - 1} - {y_n}^2\right)$ is isomorphic to the polynomial ring $F[y_n]$.  Then show any polynomial relation $y_n$ satisfies in $R_n$ gives a corresponding relation in $S_N$ for some $N \geq n$.]
            
            \subsubsection{Solution}
            
        
        \subsection{c.}
        Prove that the ideal generated by $x, x^\frac{1}{2}, x^\frac{1}{4}, ...$ in $R$ is not finitely generated (so $R$ is not a P.I.D.).
            
            \subsubsection{Solution}
            
    
    \section{9.3.1}
    Let $R$ be an integral domain with quotient field $F$ and let $p(x)$ be a monic polynomial in $R[x]$.  Assume that $p(x) = a(x)b(x)$ where $a(x)$ and $b(x)$ are monic polynomials in $F[x]$ of smaller degree than $p(x)$.  Prove that if $a(x) \notin R[x]$ then $R$ is not a Unique Factorization Domain.  Deduce that $\mathbb{Z}\left[2\sqrt{2}\right]$ is not a U.F.D.
        
        \subsection{Solution}
        
    
    \section{9.3.2}
    Prove that if $f(x)$ and $g(x)$ are polynomials with rational coefficients whose product $f(x)g(x)$ has integer coefficients, then the product of any coefficient of $g(x)$ with any coefficient of $f(x)$ is an integer.
        
        \subsection{Solution}
        
    
    \section{9.3.3}
    Let $F$ be a field.  Prove that the set $R$ of polynomials in $F[x]$ whose coefficient of $x$ is equal to 0 is a subring of $F[x]$ and that $R$ is not a U.F.D.  [Show that $x^6 = (x^2)^3 = (x^3)^2$ gives two distinct factorizations of $x^6$ into irreducibles.]
        
        \subsection{Solution}
        
    
    \section{9.3.4}
    Let $R = \mathbb{Z} + x \mathbb{Q}[x] \subset \mathbb{Q}[x]$ be the set of polynomials in $x$ with rational coefficients whose constant term is an integer.
        
        \subsection{a.}
        Prove that $R$ is an integral domain and its units are $\pm 1$.
            
            \subsubsection{Solution}
            
        
        \subsection{b.}
        Show that the irreducibles in $R$ are $\pm p$ where $p$ is a prime in $\mathbb{Z}$ and the polynomials $f(x)$ that are irreducibles in $\mathbb{Q}[x]$ \sout{and} have constant term $\pm 1$.  Prove that these irreducibles are prime in $R$.
            
            \subsubsection{Solution}
            
        
        \subsection{c.}
        Show that $x$ cannot be written as the product of irreducibles in $R$ (in particular, $x$ is not irreducible) and conclude that $R$ is not a U.F.D.
            
            \subsubsection{Solution}
            
        
        \subsection{d.}
        Show that $x$ is not prime in $R$ and describe the quotient ring $R/(x)$.
            
            \subsubsection{Solution}
            
    
    \section{13.9}
    Prove Remark 13.9: If $a, b, a', b' \in R$ are such that $a \sim a'$ and $b \sim b'$, then $ab \sim a'b'$.
        
        \subsection{Solution}
        
    
\end{document}
