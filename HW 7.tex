\documentclass[fleqn]{article}
\usepackage[margin=1in]{geometry}
\usepackage[utf8]{inputenc}
\usepackage{ulem}
\usepackage{mathtools}
\usepackage{amsmath}
\usepackage{amsthm}
\usepackage{amssymb}
\usepackage{mathabx}

\title{
Math GU4042, Spring 2020 \\
Modern Algebra II, Prof.\ Yihang Zhu \\
HW 7 \\
D\&F: 9.2.12, 9.3.\{1-4\} \\
Notes: 13.9 \\
}
\author{Khyber Sen}
\date{3/19/2020}

\setcounter{secnumdepth}{0}

\begin{document}
    
    \maketitle
    
    \section{9.2.12}
    Let $F[x, y_1, y_2, ...]$ be the polynomial ring in the infinite set of variables $x, y_1, y_2, ...$ over the field $F$, and let $I$ be the ideal $\left(x - {y_1}^2, y_1 - {y_2}^2, ..., y_i - {y_{i + 1}}^2, ...\right)$ in this ring.  Define $R$ to be the ring $F[x, y_1, y_2, ...]/I$, so that in $R$ the square of each $y_{i + 1}$ is $y_i$ and ${y_1}^2 = x$ modulo $I$, i.e., $x$ has a $2^i$th root, for every $i$.  Denote the image of $y_i$ in $R$ as $x^{\frac{1}{2^i}}$.  Let $R_n$ be the subring of $R$ generated by $F$ and $x^{\frac{1}{2^n}}$.
        
        \subsection{a.}
        Prove that $R_1 \subseteq R_2 \subseteq ...$ and that $R$ is the union of all $R_n$, i.e., $R = \bigcup\limits_{n = 1}^{\infty} R_n$.
            
            \subsubsection{Solution}
            Let
            \begin{align}
                y_0 &= x \\
                P_n &= F[y_0, ..., y_n] \\
                P &= P_\infty \\
                I_n &= \left(y_i - {y_{i + 1}}^2 \mid i \in \mathbb{N}, i < n\right) \\
                I &= I_\infty \\
                R &= P/I \\
                R_n &= \left(F, x^{\frac{1}{2^n}}\right) = (F, \pi(y_n)) \leq R
            \end{align}
            I think the problem meant to say ``Let $R_n$ be the subring of $R$ generated by \textbf{the image of F} and $x^\frac{1}{2^n}$''.  Thus, I'll instead let $R_n = \left(\pi(F), x^{\frac{1}{2^n}}\right) = (\pi(F), \pi(y_n)) \leq R$, where $\pi: P \to P/I = R$ is the canonical projection.
            
            We want to show that $R_n \subseteq R_{n + 1}$ $\forall$ $n \in \mathbb{N}$ and that $R = \bigcup\limits_{n = 1}^{\infty} R_n$.  For $R_n \subseteq R_{n + 1}$, $\pi(F) \subseteq R_{n + 1}$ is trivial.  We also know that $x_n = {x_{n + 1}}^2$, so $\pi(x_n) = \pi\left({x_{n + 1}}^2\right) \in R_{n + 1}$.  Thus, $R_n \subseteq R_{n + 1}$.
            
            For $R = \bigcup\limits_{n = 1}^{\infty} R_n$, let $\pi(p) \in R$, where the polynomial $p \in P$.  Now let $m$ be the maximum index of all of the $y_i$s in $p$.  By repeatedly substituting $y_i = {y_{i + 1}}^2$, we can write all $y_i$s with $i < m$ as powers of $y_m$.  Thus, we have that $\pi(p) \in R_m$.  Thus, every element of $R$ is in some $R_n$, so $R \subseteq \bigcup\limits_{n = 1}^{\infty} R_n$.  $\supseteq$ is trivial, so $R = \bigcup\limits_{n = 1}^{\infty} R_n$.
        
        \subsection{b.}
        Prove that $R_n$ is isomorphic to a polynomial ring in one variable over $F$, so that $R_n$ is a P.I.D.  Deduce that $R$ is a Bezout Domain.  [First show that the ring $S_n = F[x, y_1, ..., y_n]/\left(x - {y_1}^2, y_1 - {y_2}^2, ..., y_{n - 1} - {y_n}^2\right)$ is isomorphic to the polynomial ring $F[y_n]$.  Then show any polynomial relation $y_n$ satisfies in $R_n$ gives a corresponding relation in $S_N$ for some $N \geq n$.]
            
            \subsubsection{Solution}
            Let $S_n = P_n/I_n$.  We want to show that $S_n \cong F[y_n]$ and that $R_n \cong S_N$ for some $N \geq n$.  Thus, we'll have $R_n \cong F[y_N]$.  Since $F[x]$ is a PID when $F$ is a field, this will mean that $R$ is a PID, and thus $R$ is also a Bezout Domain, since all PIDs are Bezout Domains.
            
            To show $S_n \cong F[y_n]$, define $f: F[y_n] \to S_n = \pi \circ g$, where $g: F[y_n] \to F[y_0, ..., y_n]$.  Define $g$ by just extending $y_n \mapsto y_n$ homomorphically.  $g$ is clearly injective, and by rewriting all the $y_i$s in terms of $y_n$ by $y_i = {y_{i + 1}}^2$, we know it's also surjective.  Thus, it's a ring isomorphism and $S_n \cong F[y_n]$.
            
            To show $S_n \cong R_n$, define $f: S_n \to R_n = \pi \circ g \circ \pi^{-1}$, where $g: F[y_1, ..., y_n] \to (F, y_n)$.
        
        \subsection{c.}
        Prove that the ideal generated by $x, x^\frac{1}{2}, x^\frac{1}{4}, ...$ in $R$ is not finitely generated (so $R$ is not a P.I.D.).
            
            \subsubsection{Solution}
            Let $J \leq R = \left(x, x^\frac{1}{2}, x^\frac{1}{4}, ...\right) = \left(x^\frac{1}{2^n} \mid n \in \mathbb{N}\right) = (\pi(y_n) \mid n \in \mathbb{N})$.  Assume that $J$ is finitely generated, i.e., $\exists$ a finite set $A \subseteq P$ s.t.\ $(\pi(A)) = J$.  Let $a_i$ be the elements of $A$.  Note that none of the $a_i$ have a non-zero constant term since $I$ doesn't.
            Let $m$ be the maximum index of all of the $y_i$s in the elements of $A$.  Like in (a), we can keep substituting $y_i = {y_{i + 1}}^2$ until we rewrite all the $y_i$s as $y_{m + 1}$s.  Since we go one past the maximal $m$, this ensures that each $a_i = {y_{m + 1}}^k$, where $k \geq 2$.  Therefore, every element of $J$ is divisible by $\overline{{y_{m + 1}}^2}$ (where the bar is the same as $\pi$).  For $\overline{y_{m + 1}}$, this means $\overline{y_{m + 1}} = \overline{{y_{m + 1}}^2 x}$ for some $x$.  $R$ is a domain, so we can cancel out the $\overline{y_{m + 1}}$, leaving $\overline{1} = \overline{y_{m + 1} x}$.  Thus, $1 + y_{m + 1} x \in I$, but $I$ has no elements with non-zero constant terms, so this is a contradiction.  Thus, $J$ is not finitely generated.
    
    \section{9.3.1}
    Let $R$ be an integral domain with quotient field $F$ and let $p(x)$ be a monic polynomial in $R[x]$.  Assume that $p(x) = a(x)b(x)$ where $a(x)$ and $b(x)$ are monic polynomials in $F[x]$ of smaller degree than $p(x)$.  Prove that if $a(x) \notin R[x]$ then $R$ is not a Unique Factorization Domain.  Deduce that $\mathbb{Z}[2\sqrt{2}]$ is not a U.F.D.
        
        \subsection{Solution}
        Assume $R$ is a UFD.  Then, by Gauss' Lemma (Prop. 5), since $p(x)$ is reducible in $F[x]$ into $p(x) = a(x)b(x)$, then $p(x)$ is also reducible in $R[x]$, i.e., $\exists$ $r, s \in F$ s.t.\ $ra(x), sb(x) \in R[x]$ and $p(x) = ra(x) sb(x)$ is a factorization in $R[x]$.  $p(x) = rs a(x)b(x) = rs p(x)$, so $rs = 1$, i.e., $s = r^{-1}$.  Since $b(x)$ is monic, the leading coefficient of $r^{-1} b(x)$ is $r^{-1} \cdot 1 = s$, and since $r^{-1}b(x) \in R[x]$, $r^{-1} \in R$.  Thus, $r \in R$, too.  $a(x) = r^{-1} ra(x)$, and since $ra(x) \in R[x]$ and $r^{-1} \in R \subset R[x]$, $a(x) \in R[x]$.  This is a contradiction, so therefore, $R$ is not a UFD.
        
        To show that $\mathbb{Z}[2\sqrt{2}]$ is not a UFD, we can use the above proof.  Let $R = \mathbb{Z}[2\sqrt{2}]$.  Then $F = \mathbb{Q}\left[2\sqrt{2}\right]$.  Then let $p(x) = x^2 - 2 \in R[x]$, a monic polynomial, and let $a(x) = x + \sqrt{2} \in F[x]$ and $b(x) = x - \sqrt{2} \in F[x]$, both monic polynomials as well.  Thus, $a(x)b(x) = (x + \sqrt{2})(x - \sqrt{2}) = x^2 - 2 = p(x)$.  But $a(x) \notin R[x]$ because $a(x) = 1 x + \frac{1}{2} 2 \sqrt{2}$.  $1 \in \mathbb{Z}[2\sqrt{2}]$ because $1 \in \mathbb{Z}$, but $\frac{1}{2} 2 \sqrt{2} \notin \mathbb{Z}[2\sqrt{2}]$ because $\frac{1}{2} \notin \mathbb{Z}$.  Thus, $a(x) \notin R[x]$, so $R = \mathbb{Z}[2\sqrt{2}]$ is not a UFD.
    
    \section{9.3.2}
    Prove that if $f(x)$ and $g(x)$ are polynomials with rational coefficients whose product $f(x)g(x)$ has integer coefficients, then the product of any coefficient of $g(x)$ with any coefficient of $f(x)$ is an integer.
        
        \subsection{Solution}
        Since $\mathbb{Z}$ is a UFD, we can apply Gauss' Lemma to show $\exists$ $r, s \in \mathbb{Q}$ s.t.\ $rf(x), sg(x) \in \mathbb{Z}[x]$ and $rf(x) sg(x) = f(x)g(x)$.  Thus, $rs = 1$, so $s = r^{-1}$.  $\forall$ coefficients $a$ of $f(x)$ and $b$ of $g(x)$, $ra \in \mathbb{Z}$ and $sb \in \mathbb{Z}$.  Thus, $ab = a\left(r r^{-1}\right)b = (ra)(r^{-1}b) = (ra)(sb) \in \mathbb{Z}$.
    
    \section{9.3.3}
    Let $F$ be a field.  Prove that the set $R$ of polynomials in $F[x]$ whose coefficient of $x$ is equal to 0 is a subring of $F[x]$ and that $R$ is not a U.F.D.  [Show that $x^6 = (x^2)^3 = (x^3)^2$ gives two distinct factorizations of $x^6$ into irreducibles.]
        
        \subsection{Solution}
        To show that $R$ is a subring of $F[x]$, we just need to show that $1 \in R$, $R$ is closed under subtraction, and that $R$ is closed under multiplication.  $1 = 0x + 1$, so therefore, $1 \in R$.  Polynomials add term-by-term, so if the coefficients of $x$ are 0, then their difference is also 0, so $R$ is closed under subtraction.  For multiplication, the coefficient of $x$ is given by $\sum\limits_{i + j = 1} a_i b_j = a_0 b_1 + a_1 b_0 = a_0 \cdot 0 + 0 \cdot b_0 = 0$, so $R$ is also closed under multiplication, meaning $R$ is a subring of $F[x]$.
        
        To show that $R$ is not a UFD, we can just show a counterexample, i.e., a case when an element of $R$ has multiple distinct factorizations into irreducibles.  For example, consider $x^6$.  $x^ = (x^2)^3 = (x^3)^2$, and $x^2$ and $x^3$ are both irreducible in $R$ ($x^2 = x x$ and $x^3 = x^2 x$ are both invalid since $x \notin R$), so these are distinct factorizations, meaning $R$ is not a UFD.
    
    \section{9.3.4}
    Let $R = \mathbb{Z} + x \mathbb{Q}[x] \subset \mathbb{Q}[x]$ be the set of polynomials in $x$ with rational coefficients whose constant term is an integer.
        
        \subsection{a.}
        Prove that $R$ is an integral domain and its units are $\pm 1$.
            
            \subsubsection{Solution}
            $\mathbb{Q}[x]$ is a domain and $R$ is a subring of $\mathbb{Q}[x]$, so $R$ is also a domain.  Since $R$ is a subring of $\mathbb{Q}[x]$, $R^\times \subseteq \mathbb{Q}[x]^\times$.  $\mathbb{Q}$ is a domain, so $\mathbb{Q}[x]^\times = \mathbb{Q}^\times$.  That is, the units are all constant polynomials.  In $R$, constant polynomials are all integers, so $R^\times = \mathbb{Z}^\times = \{\pm 1\}$.
        
        \subsection{b.}
        Show that the irreducibles in $R$ are $\pm p$ where $p$ is a prime in $\mathbb{Z}$ and the polynomials $f(x)$ that are irreducibles in $\mathbb{Q}[x]$ and have constant term $\pm 1$.  Prove that these irreducibles are prime in $R$.
            
            \subsubsection{Solution}
            Let $p(x) \in R$ be irreducible.  We want to show that $p(x) = \pm p$, where $p$ is a prime, or $p(x)$ is irreducible in $\mathbb{Q}[x]$ and $p(x)$ has a constant term of $\pm 1$.  If $\deg(p(x)) = 0$, then $p(x) = p$ is a constant polynomial and $p \in \mathbb{Z}$.  Since $\mathbb{Z}$ is a PID and irreducibles are prime in a PID, $p$ must then be prime since $p$ is irreducible.
            
            Otherwise, $\deg(p(x)) > 0$ ($\deg(p(x)) < 0$ is impossible because $p(x) = 0$ since $0$ is not irreducible by definition).  Now assume that $p(x)$ is reducible in $\mathbb{Q}[x]$, i.e., $p(x) = a(x)b(x)$, where $a(x), b(x) \in \mathbb{Q}[x]$.  Furthermore, $p(x) = q p'(x)$, where $q \in \mathbb{Q}$ and $p'(x) \in \mathbb{Z}[x]$.  Therefore, $p'(x) = q^{-1} a(x)b(x)$ is reducible in $\mathbb{Q}[x]$.  Applying Gauss' Lemma, we know that $p'(x) = a'(x) b'(x)$, where $a'(x), b'(x) \in \mathbb{Z}[x]$.  $p(x) = q a'(x) b'(x)$.  If we look at just the constant term, which is an integer, we know that $q a_0 b_0$ is also an integer, where $a_0$ and $b_0$ are the integer constant terms.  $q \in \mathbb{Q}$, however, so write $q = \frac{z}{xy}$, where $x \mid a_0$ and $y \mid b_0$.  Thus, $p(x) = \left(\frac{z}{x} a(x)\right)\left(\frac{1}{y} b(x)\right)$ is a factorization of $p(x)$ in $R$, meaning $p(x)$ is reducible in $R$.  This is a contradiction, so $p(x)$ must be irreducible in $\mathbb{Q}[x]$.
            
            Now, to show the constant term is always $\pm 1$, let $p(x) = a + p'(x) x$.  Assume $a \notin \{\pm 1\}$.  Then $\exists$ a prime $q$ s.t.\ $q \mid a$, i.e.\ $\exists$ $b$ s.t.\ $a = qb$.  Therefore, $p(x) = q\left(b + \frac{1}{q} p'(x) x\right)$.  $q$ is not a unit, and neither is $b + \frac{1}{q} p'(x) x$, so therefore, $p(x)$ is reducible, which is a contradiction.  Thus, $a \in \{\pm 1\}$.
        
        \subsection{c.}
        Show that $x$ cannot be written as the product of irreducibles in $R$ (in particular, $x$ is not irreducible) and conclude that $R$ is not a U.F.D.
            
            \subsubsection{Solution}
            Assume $x$ can be written as the product of irreducibles in $R$, i.e., $x = \prod a_i$, where $a_i$ are irreducibles.  $1 = \deg(x) = \deg(\prod a_i) = \sum \deg(a_i)$, so exactly one $a_i$ has degree 1 and the rest are degree 0.  WLOG, let $a_0$ be the degree 1 factor.  By (b), $a_0$ has a constant term of $\pm 1$.  When multiplied by all the other constant $a_i$s, this gives a non-zero constant term for $\prod a_i = x$.  But $x$'s constant term is 0, which is a contradiction.  Thus, $x$ cannot be written as the product of irreducibles in $R$.  Since $x$ is non-zero and non-unit in $R$ but it can't be factored, $R$ must not be a UFD.
        
        \subsection{d.}
        Show that $x$ is not prime in $R$ and describe the quotient ring $R/(x)$.
            
            \subsubsection{Solution}
            $x = 2 \cdot \frac{1}{2} x$ and $2, \frac{1}{2} x \in R$.  Assume $x$ is prime in $R$, so since $x \mid 2 \cdot \frac{1}{2} x$, $x \mid 2$ or $x \mid \frac{1}{2} x$.  $x \not\mid 2$, so therefore, $x \mid \frac{1}{2} x$ in $R$.  But $\frac{\frac{1}{2} x}{x} = \frac{1}{2} \notin R$.  This is a contradiction, so therefore, $x$ is not prime in $R$.
            
            For $R/(x)$, $R/(x) = \{a + bx + (x) \mid a \in \mathbb{Z}, b \in \mathbb{Q} \cap [0, 1)\}$.  The $\supseteq$ direction is trivial.  For the $\subseteq$ direction, let $p(x) + (x) \in R/(x)$.  WLOG, let $\deg(p(x)) < 2$, since higher degree terms will be cancelled out by the $(x)$.  Note that degree 1 terms won't be completely cancelled out, because for $ax \in (x)$, $a \in R$, and if $a$ is a constant in $R$, $a$ must be an integer.  Now, we need to show $p(x) = a + bx$ for some $a \in \mathbb{Z}$ and $b \in \mathbb{Q} \cap [0, 1)$.  Let $p(x) = p_1 x + p_0$.  The $(x)$ cancels out a bit more, however, but only up to integer multiples.  That is, $p_1 x = p_1' x + kx$, where $k \in \mathbb{Z}$ and $p_1' \in [0, 1)$, since $kx \in (x)$ because $k \in R$.
    
    \section{13.9}
    Prove Remark 13.9: If $a, b, a', b' \in R$ are such that $a \sim a'$ and $b \sim b'$, then $ab \sim a'b'$.
        
        \subsection{Solution}
        We need to show that $(a) = (a'), (b) = (b') \implies (ab) = (a'b')$.  $(a) = (a')$ means that $a = xa'$, where $x$ is a unit, since then $a = xa'$ and $a' = x^{-1}a$.  Likewise, $(b) = (b')$ means that $b = yb'$, where $y$ is a unit.  And to show $(ab) = (a'b')$, we need to show that $ab = za'b'$, where $z$ is a unit.  $ab = (xa')(yb') = (xy)(a'b')$.  $z = xy$ is a unit since $(xy)(y^{-1}x^{-1}) = 1$, so therefore $(ab) = (a'b')$.
    
\end{document}
