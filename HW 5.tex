\documentclass[fleqn]{article}
\usepackage[margin=1in]{geometry}
\usepackage[utf8]{inputenc}
\usepackage{ulem}
\usepackage{mathtools}
\usepackage{amsmath}
\usepackage{amsthm}
\usepackage{amssymb}
\usepackage{mathabx}

\title{
Math GU4042, Spring 2020 \\
Modern Algebra II, Prof.\ Yihang Zhu \\
HW 5 \\
D\&F: 8.\{1.\{4-7, 10\}, 2.\{4, 7.(a, b)\}, 3.11\} \\
Notes: 9.\{4, 10\} \\
}
\author{Khyber Sen}
\date{2/27/2019}

\setcounter{secnumdepth}{0}

\begin{document}
    
    \maketitle
    
    \section{8.1.4}
    Let $R$ be a Euclidean Domain.
        
        \subsection{(a)}
        Prove that if $(a, b) = 1$ and $a$ divides $bc$, then $a$ divides $c$.  More generally, show that $a$ divides $bc$ with nonzero $a, b$, then $\frac{a}{(a, b)}$ divides $c$.
            
            \subsubsection{Solution}
            
        
        \subsection{(b)}
        Consider the Diophantine Equation $ax + by = N$ where $a, b$ and $N$ are integers and $a, b$ are nonzero.  Suppose $x_0, y_0$ is a solution: $ax_0 + by_0 = N$.  Prove that the full set of solutions to this equation is given by
        \begin{align}
            x &= x_0 + m \frac{b}{(a, b)} \\
            y &= y_0 - m \frac{a}{(a, b)}
        \end{align}
        as $m$ ranges over the integers.  [If $x, y$ is a solution to $ax + by = N$, show that $a(x - x_0) = b(y_0 - y)$ and use (a).]
        
            \subsubsection{Solution}
            
    
    \section{8.1.5}
    Determine all integer solutions of the following equations:
        
        \subsection{(a)}
        $2x + 4y = 5$
            
            \subsubsection{Solution}
            
        
        \subsection{(a)}
        $17x + 29y = 31$
            
            \subsubsection{Solution}
            
        
        \subsection{(a)}
        $85x + 145y = 505$
        
            \subsubsection{Solution}
            
    
    \section{8.1.6}
    (\textit{The Postage Stamp Problem})
    
    Let $a$ and $b$ be two relatively prime positive integers.  Prove that every sufficiently large positive integer $N$ can be written as a linear combination $ax + by$ of $a$ and $b$ where $x$ and $y$ are both \textit{nonnegative}, i.e., there is an integer $N_0$ such that for all $N \geq N_0$ the equation $ax + by = N$ can be solved with both $x$ and $y$ nonnegative integers.  Prove in fact that the integer $ab - a - b$ cannot be written as a positive linear combination of $a$ and $b$ but that every integer greater than $ab - a - b$ is a positive linear combination of $a$ and $b$ (so every ``postage'' greater than $ab - a - b$ can be obtained using only stamps in denominations $a$ and $b$).
        
        \subsection{Solution}
        
    
    \section{8.1.7}
    Find a generator for the ideal $(85, 1 + 13i)$ in $\mathbb{Z}[i]$, i.e., a greatest common divisor for 85 and $1 + 13i$, by the Euclidean Algorithm.  Do the same for the ideal $(47 - 13i, 53 + 56i)$.
        
        \subsection{Solution}
        
    
    \section{8.1.10}
    Prove that the quotient ring $\mathbb{Z}[i]/I$ is finite for any nonzero ideal $I$ of $\mathbb{Z}[i]$.  [Use the fact that $I = (\alpha)$ for some nonzero $\alpha$ and then use the Division Algorithm in this Euclidean Domain to see that every coset of $I$ is represented by an element of norm less than $N(\alpha)$.]
        
        \subsection{Solution}
        
    
    \section{8.2.4}
    Let $R$ be an integral domain.  Prove that if the following two conditions hold then $R$ is a Principle Ideal Domain:
        
        \subsection{(i)}
        any two nonzero elements $a$ and $b$ in $R$ have a greatest common divisor which can be written in the form $ra + sb$ for some $r, s \in R$, and
        
        \subsection{(ii)}
        if $a_1, a_2, a_3, ...$ are nonzero elements of $R$ such that $a_{i + 1} \mid a_i$ for all $i$, then there is a positive integer $N$ such that $a_n$ is unit times $a_N$ for all $n \geq N$.
        
        \subsection{Solution}
        
    
    \section{8.2.7.(a, b)}
    An integral domain $R$ in which every ideal generated by two elements is principal (i.e., for every $a, b \in R$, $(a, b) = (d)$ for some $d \in R$) is called a \textit{Bezout Domain}.
        
        \subsection{(a)}
        Prove that the integral domain $R$ is a Bezout Domain if and only if every pair of elements $a, b$ of $R$ has a g.c.d. $d$ in $R$ that can be written as an $R$-linear combination of $a$ and $b$, i.e., $d = ax + by$ for some $x, y \in R$.
            
            \subsubsection{Solution}
            
        
        \subsection{(b)}
        Prove that every finitely generated ideal of a Bezout Domain is principal.
        
            \subsubsection{Solution}
            
    
    \section{8.3.11}
    (\textit{Characterization of P.I.D.s})
    
    Prove that $R$ is a P.I.D. if \sout{and only if} $R$ is a U.F.D. that is also a Bezout Domain.  [One direction is given by Theorem 14.  For the converse, let $a$ be a nonzero element of the ideal $I$ with a minimal number of irreducible factors.  Prove that $I = (a)$ by showing that if there is an element $b \in I$ that is not in $(a)$ then $(a, b) = (d)$ leads to a contradiction.]
        
        \subsection{Solution}
        
    
    \section{9.4}
    Check that $~$ is an equivalence relation on $R$.  Also check that the set of equivalence classes in $R$ is in bijection with the set of principal ideals in $R$.  (Warning: a principal ideal $(a)$, however, is not equal to the equivalence class of $a$.)
        
        \subsection{Solution}
        
    
    \section{9.10}
    Check all the statements in the above example.
    
        \subsection{Example 9.9}
        Consider $a = 2, b = X \in R = \mathbb{Z}[X]$.  Then 1 is a gcd of $a, b$ (and hence $\pm 1$ are all the gcd's because $R^\times = \mathbb{Z}^\times = \{\pm 1\}$).  However, 1 is not a Bgcd, because $(2, X) \neq 1$.
        
        \subsection{Solution}
        
    
\end{document}
